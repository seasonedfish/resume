\section{Education}
% doutline means Description Outline, used to bold the text inside brackets
\begin{doutline}
    % We use a \ before the space to tell Latex that the . in St. doesn't end a sentence.
    % https://tex.stackexchange.com/a/2230
    \1[Washington University in St.\ Louis] GPA 3.94/4.00
    \hfill May 2025 (expected)
        \2 Bachelor of Science in Computer Science
        \2 Relevant Courses: Data Structures \& Algorithms, Rapid Prototype Development, Intro to Data Science
\end{doutline}

\section{Skills}
% labeling allows for a two-column outline with the left side bolded.
\begin{labeling}{Programming Languages}
    \item [Programming Languages] Python, Java, Bash, C, C++, Haskell, PHP, R
    \item [Markup Languages] \LaTeX, HTML, Markdown, reStructuredText
    \item [Libraries] Pandas, Matplotlib, Gson, Apache Commons CLI, Polars, Beautiful Soup
    \item [Tools] Git, JUnit, Pytest, Maven, Poetry, GitHub Actions, Coveralls, Nix, GNU Parallel
    \item [Operating Systems] Windows, macOS, Linux (KDE Plasma), UNIX command line
\end{labeling}

\section{Experience}
\begin{doutline}
    % We use hfill to align the year to the right.
    \1[UMass Lowell] / Research Assistant
    \hfill May 2022--July 2022
        \2 Developed optimized, parallel Python code to transform dataset with over 100 million entries
        \2 Automated the multithreaded downloading of thousands of files using ThreadPoolExecutor
        \2 Extracted relevant data from over 2 GB of SEC S-1 documents using Python
    \1[UMass Chan Medical School] / Research Assistant
    \hfill July 2020--Sept.\ 2020
        \2 Designed preprocessing module using Pandas to standardize information from multiple datasets
            \3 Developed idiomatic, efficient code using vectorized functions
            \3 Used Pytest for integration testing and GitHub Actions for continous integration (CI)
    % We use two dashes (--) to create an en-dash for the year range.
    % https://tex.stackexchange.com/a/3821
    \1[Energize Andover] / Software Engineering Intern
    \hfill July 2019--Oct.\ 2021
        \2 Developed a Python client for Building Energy Gateway using Requests
            \3 Reached 100\% code coverage using Coveralls and Pytest for unit testing
        \2 Created an Alexa app with Flask and AWS Lambda to track the temperature of Andover High School
\end{doutline}

\section{Personal Projects}
\begin{doutline}
    \1[Pokésummary] / Python, Poetry, Pytest, GitHub Actions
    \hfill Feb.\ 2021--Present
        \2 Developed command-line program that can display information about any of 1000+ Pokémon
            \3 Followed Model-View-Controller design pattern for program structure
            \3 Tested using Pytest, packaged using Poetry, published to the Python Package Index (PyPI)
        \2 Maintaining code and CI to ensure compatibility across Python 3.7, 3.8, 3.9, and 3.10
    \1[Coding Blog] / Jekyll, HTML + Liquid, SCSS, Netlify
    \hfill Nov.\ 2020--Present
        \2 Built a static site for personal coding reflections using Jekyll, deployed it using Netlify
            \3 Created a custom theme with HTML + Liquid, SCSS
    \1[File-hosting site] / Amazon EC2, Amazon Linux 2, Apache, PHP, HTML, CSS
    \hfill Aug.\ 2022--Sept.\ 2022
        \2 Developed a file-hosting site with PHP, HTML, and CSS supporting multiple users; each user has a quota
        \2 Set up an EC2 instance running Amazon Linux 2 and Apache to deploy it
\end{doutline}

\section{Activities}
\begin{doutline}
    \1[Developer Student Club WashU] / Technical Lead
    \hfill Oct.\ 2021--Present
        \2 Teaching members Git and coding skills through workshops and help channels
\end{doutline}
